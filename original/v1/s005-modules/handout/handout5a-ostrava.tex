% arara: lualatex
% arara: lualatex
% Typeset using lualatex

\documentclass[a4paper,10pt]{article}
\usepackage[utf8]{luainputenc}
\usepackage[pdfborder={0 0 0}]{hyperref}

\usepackage[czech]{babel}

\usepackage{fouriernc}
\usepackage{tgpagella}
\usepackage{xcolor}
\usepackage[activate=true]{microtype}
\usepackage{amssymb}
\usepackage{fontspec}
\usepackage{titling}
\usepackage{titlesec}

\usepackage{calc}
\usepackage{multicol}
\usepackage{array}
\usepackage{enumitem}
\usepackage{tikz}
\usepackage{siunitx}
\usepackage{colortbl}
\usepackage{everypage}
\usepackage[a4paper,margin=2cm,top=1cm]{geometry}

\setmainfont[Numbers=OldStyle]{TeX Gyre Pagella}

\definecolor{plpink}{HTML}{FF3232}
\definecolor{plhome}{HTML}{CCEEFF}
\definecolor{silver}{HTML}{888888}

\sisetup{
    output-decimal-marker={,}
}
\pagestyle{empty}

\parskip=1ex
\parindent=0pt

\setlist[enumerate,1]{start=0}

\AddEverypageHook{
    \begin{tikzpicture}[remember picture,overlay]
        \node[opacity=0.1,below left,xshift=1cm,yshift=1cm] at (current page.north east) {\includegraphics[width=12cm]{../../images/pylady-pink}};
        \node[below left,align=right,yshift=-1.5em] at (current page.north east) {\color{white} strana \thepage};
        \node[below left,align=right] at (current page.north east) {\color{white} sada \plsetno};
    \end{tikzpicture}
}

\newcommand\answerspace{\\\rule[0cm]{0pt}{1cm}}

\newcommand\True{\texttt{True}}
\newcommand\False{\texttt{False}}

\usetikzlibrary{turtle}
\newcommand\turtle[1]{\tikz[>=stealth,x=0.25mm,y=0.25mm]\draw[turtle={home,rt=90,#1}];}
\newcommand\gulp[1]{}

\newcommand\plsetno{6}

\newcommand\startsection[1]{
     \vspace{0.2ex}
    \hrule
    {\fontspec{Oxygen} \tiny
     \vspace{-1ex}
     \emph{#1}
     \vspace{-1.5em}
    }
}

\newfontfamily\headingfont[]{Bree Serif}
\titleformat*{\section}{\LARGE\headingfont}

\begin{document}

\section*{Domácí projekty \plsetno}

\startsection{Máš-li hotové Piškvorky, můžeš si naprogramovat ještě jednu hru! Jestli na to tedy máš čas a chuť.}

\begin{enumerate}[resume]

\item Vytvoř hru \texttt{sibenice} podle následujícího popisu.
    Snaž se projekt rozdělit do funkcí s hezkými jmény, piš dokumentační řetězce.

    \begin{itemize}
        \item Počítač náhodně zvolí slovo (zatím třeba ze tří možností).
            Pro jednoduchost používej malá písmena a nepoužívej slova,
            ve kterých se stejné písmeno opakuje dvakrát
            (třeba č\textcolor{plpink}{o}k\textcolor{plpink}{o}láda).
        \item Nastav výchozí stav: řetězec s tolika podtržítky, kolik je
             ve vybraném slově písmen.
        \item Nastav počet neúspěšných pokusů na nulu
        \item Každý tah:
            \begin{itemize}
                \item Zeptej se hráče na písmeno.
                \item Pokud je to písmeno ve vybraném slově, zaměň příslušná
                    podtržítka za ono písmeno.
                    \\\emph{\small Bude se hodit řetězcová metoda \texttt{index}
                    a funkce \texttt{zamen} z piškvorek.}
                \item Pokud dané písmeno \emph{není} ve vybraném slově,
                    započítej neúspěšný pokus.
                \item Vypiš stav (slovo s případnými podtržítky).
                \item Pokud už ve slově nejsou podtržítka, pogratuluj hráči
                    a ukonči hru.
                \item Podle počtu neúspěšných pokusů vypiš „obrázek”.
                    Obrázky jsou ke stažení na \\\url{http://pyladies.cz/v1/s005-modules/obrazky.txt}
                    a můžeš je dát do víceřádkových řetězců (s trojitými uvozovkami).
                \item Pokud jsi právě vypsala poslední obrázek, hráč prohrál.
                    Dej mu to najevo a ukonči program.
            \end{itemize}
    \end{itemize}

\begin{verbatim}
+---.
|   |
|   O
| --|--
|  /
|
~~~~~~~

p_t__n

\end{verbatim}

\item Funguje-li ti předchozí hra, můžeš ji vylepšit.

    \begin{itemize}
        \item Zařiď, aby fungovala slova s více stejnými písmeny.
        \item Když hráč nezadá písmeno (zadá např. \texttt{ABC} nebo \texttt{!}),
            nepovažuj to za tah.
        \item Po skončení dej hráči možnost hru opakovat.
    \end{itemize}

\end{enumerate}

\end{document}
