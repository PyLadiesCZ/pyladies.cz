% arara: lualatex
% arara: lualatex
% Typeset using lualatex

\documentclass[a4paper,10pt]{article}
\usepackage[utf8]{luainputenc}
\usepackage[pdfborder={0 0 0}]{hyperref}

\usepackage[czech]{babel}

\usepackage{fouriernc}
\usepackage{tgpagella}
\usepackage{xcolor}
\usepackage[activate=true]{microtype}
\usepackage{amssymb}
\usepackage{fontspec}
\usepackage{titling}
\usepackage{titlesec}

\usepackage{calc}
\usepackage{multicol}
\usepackage{array}
\usepackage{enumitem}
\usepackage{tikz}
\usepackage{siunitx}
\usepackage{colortbl}
\usepackage{everypage}
\usepackage[a4paper,margin=2cm,top=1cm]{geometry}

\setmainfont[Numbers=OldStyle]{TeX Gyre Pagella}

\definecolor{plpink}{HTML}{FF3232}
\definecolor{plhome}{HTML}{CCEEFF}
\definecolor{silver}{HTML}{888888}

\sisetup{
    output-decimal-marker={,}
}
\pagestyle{empty}

\parskip=1ex
\parindent=0pt

\setlist[enumerate,1]{start=0}

\AddEverypageHook{
    \begin{tikzpicture}[remember picture,overlay]
        \node[opacity=0.1,below left,xshift=1cm,yshift=1cm] at (current page.north east) {\includegraphics[width=12cm]{../../images/pylady-pink}};
        \node[below left,align=right,yshift=-1.5em] at (current page.north east) {\color{white} strana \thepage};
        \node[below left,align=right] at (current page.north east) {\color{white} sada \plsetno};
    \end{tikzpicture}
}

\newcommand\answerspace{\\\rule[0cm]{0pt}{1cm}}

\newcommand\True{\texttt{True}}
\newcommand\False{\texttt{False}}

\usetikzlibrary{turtle}
\newcommand\turtle[1]{\tikz[>=stealth,x=0.25mm,y=0.25mm]\draw[turtle={home,rt=90,#1}];}
\newcommand\gulp[1]{}

\newcommand\plsetno{9}

\newcommand\startsection[1]{
     \vspace{0.2ex}
    \hrule
    {\fontspec{Oxygen} \tiny
     \vspace{-1ex}
     \emph{#1}
     \vspace{-1.5em}
    }
}

\newfontfamily\headingfont[]{Bree Serif}
\titleformat*{\section}{\LARGE\headingfont}

\begin{document}

\section*{Domácí projekty \plsetno}

\startsection{Těchto pár úkolů slouží k procvičení práce se slovníky, které se
    následně pokusíš využít při programování známé skautské hry.}

\begin{enumerate}

\item Napiš funkci, která pro argumentem zadané číslo \texttt{n} vytvoří a vrátí slovník,
    kde jako klíče budou čísla od jedné do \texttt{n} a jako hodnoty k nim
    jejich druhé mocniny.

\item Napiš funkci, která sečte a vrátí sumu všech klíčů a sumu všech hodnot
    ve slovníku, který dostane jako argument. K vyzkoušení můžeš použít
    slovník z minulé úlohy.

\item Napiš funkci, která jako argument dostane řetězec a vrátí slovník,
    ve kterém budou jako klíče jednotlivé znaky ze zadaného řetězce a jako
    hodnoty počty výskytů těchto znaků v řetězci.

\item Napiš funkci, která vypíše obsah slovníku (klíče a k nim náležící
    hodnoty) na jednotlivé řádky.

    \emph{Funkci, která něco vypisuje, je vhodné
    pojmenovat speciálně, zde třeba \texttt{vypis\_slovnik}, aby bylo jasné,
    že jen vypisuje a nic nevrací.}

\end{enumerate}

\startsection{A teď už ke slíbené hře.}

\begin{enumerate}[resume]

\item Úkolem je vytvořit známou skautskou hru „Kdo? S kým? Co dělali?“. Hra se
    hráče zeptá postupně na různé otázky, například „Kdo?“, „S kým?“,
    „Co dělali?“, „Kde?“, „Kdy?“, a nakonec „Proč?“, s tím, že mu umožní
    na jednu otázku odpovědět vícekrát a všechny odpovědi si uloží.
    Na závěr pak hra z každé sady odpovědí vybere náhodně jednu odpověď
    a z takto vybraných odpovědí složí větu, kterou vypíše uživateli.
    Seznam otázek by mělo být možné změnit na jednom místě bez zásahu
    do logiky programu.

\end{enumerate}

\end{document}
