% Typeset using lualatex

\documentclass[a4paper,10pt]{article}
\usepackage[utf8]{luainputenc}
\usepackage[pdfborder={0 0 0}]{hyperref}

\usepackage[czech]{babel}

\usepackage{fouriernc}
\usepackage{tgpagella}
\usepackage{xcolor}
\usepackage[activate=true]{microtype}
\usepackage{amssymb}
\usepackage{fontspec}
\usepackage{titling}
\usepackage{titlesec}

\usepackage{calc}
\usepackage{multicol}
\usepackage{array}
\usepackage{enumitem}
\usepackage{tikz}
\usepackage{siunitx}
\usepackage{colortbl}
\usepackage{everypage}
\usepackage[a4paper,margin=2cm,top=1cm]{geometry}

\setmainfont[Numbers=OldStyle]{TeX Gyre Pagella}

\definecolor{plpink}{HTML}{FF3232}
\definecolor{plhome}{HTML}{CCEEFF}
\definecolor{silver}{HTML}{888888}

\sisetup{
    output-decimal-marker={,}
}
\pagestyle{empty}

\parskip=1ex
\parindent=0pt

\setlist[enumerate,1]{start=0}

\AddEverypageHook{
    \begin{tikzpicture}[remember picture,overlay]
        \node[opacity=0.1,below left,xshift=1cm,yshift=1cm] at (current page.north east) {\includegraphics[width=12cm]{../../images/pylady-pink}};
        \node[below left,align=right,yshift=-1.5em] at (current page.north east) {\color{white} strana \thepage};
        \node[below left,align=right] at (current page.north east) {\color{white} sada \plsetno};
    \end{tikzpicture}
}

\newcommand\answerspace{\\\rule[0cm]{0pt}{1cm}}

\newcommand\True{\texttt{True}}
\newcommand\False{\texttt{False}}

\usetikzlibrary{turtle}
\newcommand\turtle[1]{\tikz[>=stealth,x=0.25mm,y=0.25mm]\draw[turtle={home,rt=90,#1}];}
\newcommand\gulp[1]{}

\newcommand\plsetno{6}

\newcommand\startsection[1]{
     \vspace{0.2ex}
    \hrule
    {\fontspec{Oxygen} \tiny
     \vspace{-1ex}
     \emph{#1}
     \vspace{-1.5em}
    }
}

\newfontfamily\headingfont[]{Bree Serif}
\titleformat*{\section}{\LARGE\headingfont}

\begin{document}

\section*{Domácí projekty \plsetno}

\startsection{Dnešní projekty jsou součástí projektu 1D Piškvorky. Dělej je jeden po druhém.}

\begin{enumerate}
\item Přidej do 1D Piškvorek nový modul \texttt{test\_piskvorky.py}, kde budou testy.

\item Napiš \emph{aspoň} dva testy na každou funkci z Piškvorek, ke které testy napsat umíš.
    \\\emph{\small Testy pravděpodobně neumíš napsat na funkci \texttt{input} a všechny funkce, které ji (byť nepřímo) volají.
            A taky asi neumíš otestovat účinek funkce \texttt{print}.}

\item Zkus program přepsat tak, aby část, která není pokrytá automatickými testy, byla co nejmenší.
    \\\emph{\small Místo volání \texttt{print} můžeš často vrátit řetězec
            a \texttt{print} zavolat až na návratovou hodnotu.
            Podobně \texttt{input} můžeš zavolat před zavoláním funkce,
            která pak může příkaz od uživatele brát jako argument.}

\item Ověř, že se funkce \texttt{tah\_pocitace} se umí vyrovnat s jinou délkou hracího pole než 20.
    \\\emph{\small „Ověření” znamená napsání příslušného testu.}

\item Ověř si, že se \texttt{tah\_pocitace} chová rozumně když dostane plné hrací pole, nebo pole s délkou 0.
    \\\emph{\small Rozumné chování v tomto případě znamená vyvolání rozumné výjimky.}

\end{enumerate}

\end{document}
